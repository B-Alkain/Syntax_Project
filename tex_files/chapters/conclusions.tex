\section{Conclusions}
\begin{itemize}
  \item Our HMM edges the \texttt{nltk} HMM in both languages: +3.6 test points in Basque (0.8560 vs. 0.8189) and +0.15 in Catalan (0.9445 vs. 0.9430), with dev gains of 0.8622 vs. 0.8258 (EU) and 0.9477 vs. 0.9453 (CA).
  \item Against backoff n-gram taggers, Basque trigram hits 0.866 (slightly above our HMM 0.8560 and well above \texttt{nltk} 0.8189); in Catalan the HMMs lead over n-grams (0.9445/0.9430 vs. 0.935). Joint modeling of transitions and emissions yields a clear advantage in CA and is on par in EU.
  \item Language effect: Basque shows larger train\(\rightarrow\)dev/test drops (0.9693\(\rightarrow\)0.8622\(\rightarrow\)0.8560) than Catalan (0.9761\(\rightarrow\)0.9477\(\rightarrow\)0.9445), reflecting how agglutinative morphology increases \(p(x\mid y)\) sparsity.
  \item Tag-wise (Basque test): PUNCT 1.000, PART 0.997, and CCONJ 0.986 lead; open classes dip (NOUN 0.853, VERB 0.823, PROPN 0.662) and rare tags drop sharply (INTJ 0.125, X 0.435, SCONJ 0.000). In Catalan, overall scores rise (accuracy 0.94479, macro-F1 0.836): closed classes stay high (ADP 0.986 F1, PUNCT 0.988 F1), and open classes improve (NOUN 0.940 F1, VERB 0.936 F1), with only INTJ and PART showing low recall (0.000 and 0.190).
  \item Confusion matrices align with these patterns: diagonals are cleaner in Catalan than Basque, and our HMM reduces off-diagonal mass compared to \texttt{nltk}, especially for frequent tags.
  \item Per-tag bars and F1 comparisons confirm gains across most tags for our HMM over NLTK in both languages, consistent with higher macro-F1 (EU 0.772 vs. 0.745 macro recall; CA 0.836 vs. 0.827 macro recall).
  \item Sequential probability: the joint probability for a Basque example was \(4.29\times 10^{-13}\), a reasonable scale for long sequences; random samples and Viterbi paths confirm the model favors well-formed tag orders.
\end{itemize}
